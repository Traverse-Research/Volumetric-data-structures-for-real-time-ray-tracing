\section*{Abstract} \label{abstract}
Volumetric effects such as clouds, explosions, smoke, and fog are important scene elements for computer games. While these can be efficiently handled in a rasterizer, path tracers typically struggle to render them efficiently. This thesis provides the required prior knowledge to think about the different trade-offs of volume data structures, and to provide a specialized implementation for compressed density data. The resulting data structure reduces the voxel data memory footprint by a factor of 8 to 16 times over 16-bit floating point values. This is done by utilizing a novel method of storing density data in block-compressed textures and deduplicating homogeneous nodes. These methods allow us to store animation sequences in game-ready asset sizes and render them at real-time frame rates.