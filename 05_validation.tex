\section{Evaluation procedure}\label{EVALUATION}
To evaluate the data structures described in Section \ref{REQUIREMENTS:concretization}, multiple data structures must be built \ref{EVALUATION:data_structures} tested against a set of scenes which cover multiple worst case scenarios \ref{EVALUATION:scenes} and compared using a set of metrics \ref{EVALUATION:metrics}.

\subsection{Data structures}\label{EVALUATION:data_structures}
The in Section \ref{REQUIREMENTS:concretization} suggested data structures will be implemented. These structures will be compared against a flat 3D array, as this is the simplest volumetric data structure, while still being actively used in many areas. Along with that, we will compare against the ray tracing performance of NanoVDB \cite{NanoVDBBenchmark}. So we end up with the following data structures:
\begin{enumerate}
    \item Dynamic animation VDB
    \item Dynamic simulation VDB
    \item Flat grid
    \item NanoVDB (only for ray tracing performance)
\end{enumerate}


\subsection{Scenes}\label{EVALUATION:scenes}
There will be a multitude of different scenes with which different metrics will be tested. These data sets will be either downloaded from \href{https://www.openvdb.org/download/}{OpenVDB's} or \href{https://jangafx.com/software/embergen/download/free-vdb-animations/}{JangaFX's} website, or be generated using procedural generation.

\noindent\textbf{Static scenes:}
\begin{enumerate}[noitemsep,topsep=0pt,parsep=0pt,partopsep=0pt]
    \item OpenVDB's bunny cloud for ray tracing performance of a smoke volume. Specifically for comparison with an existing benchmark by NVIDIA. This will be a benchmark of the traversal through a actual volumetric smoke, which will give insight in the performance of traversing through a heterogeneous volume.
    \item OpenVDB's dragon for ray tracing empty space. Again, for comparison. Unlike the bunny, this scene only finds the primary intersection. Which will highlight the difference in performance between tracing through sparse empty space, and heterogeneous volume data.
\end{enumerate}

\noindent\textbf{Animation scenes:}
\begin{enumerate}[noitemsep,topsep=0pt,parsep=0pt,partopsep=0pt]
    \item JangaFX's gasoline explosion. This scene has not only densities but also emissions, which will test FR \ref{FR:datalayout}.
    \item JangaFX's dust shockwave. This scene contains a very large animation. And can thus be used very well for comparing memory usage, and delta compression (compression applied to the difference between two frames in an animation).
\end{enumerate}

\noindent\textbf{Simulation scenes:}
\begin{enumerate}[noitemsep,topsep=0pt,parsep=0pt,partopsep=0pt]
    \item Falling sand simulation. This scene has a simple implementation with only a small area in the volume being changed every frame. The rest of the scene will be empty which allows us to benchmark the sparsity if our data structure. Along with that, this will almost be a hello world type of scene as the behavior is very predictable.
    \item A million particles expanding in random directions through the scene, which will be a worst case with as many topology changes in the tree as possible. Once these particles have moved outside of the AABB of the volume we should also see the data structure reduce to a very small size as it will not contain any data, and thus be completely sparse. 
    \item Solid heterogeneous volume with changing densities. This will stress the simulation performance without changing the topology. Additionally, the entire tree needs to be allocated which will highlight the worst case memory usage.
\end{enumerate}

\subsection{Evaluation metrics}\label{EVALUATION:metrics}
For all these scenes we will benchmark the update/simulation speed in milliseconds, the memory usage in megabytes, the primary ray traversal in million rays per second and milliseconds per frame. The simulation scenes will be at different resolutions ranging from $128^3$ to $4069^3$. Respectively this would be 2MB to 68.7GB when stored in a flat 3d grid. Results will be presented where possible as commodity graphics hardware has a maximum of 24GB of VRAM.
